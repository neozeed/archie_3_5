%
% Copyright (c) 1996 Bunyip Information Systems Inc.
% All rights reserved
%
% Archie 3.5
% August 1996
%
% appendix.tex
%


\begin{appendix}



\chapter{Header record}
\label{app:header}


The following header record fields are currently used by the Archie system:


\setlongtables
\begin{longtable}[!]{lp{4.5in}}
Field Name & Function \\ \\  \hline \\
\endhead
\endfoot

primary\_hostname & 
The primary hostname of the site to which the data belongs. These names are
used internally by the Archie system
 \\

preferred\_hostname &
The name under which users see this site listed. It will be a valid Domain
Name System canonical name (CNAME) for that site if one has been set \\

generated\_by & 
The component of the Archie system which has generated this header. Valid
values are: 
\begin{TTentry}{retrieve}
\item[parser] Output from the parse phase
\item[retrieve] Output from the data acquisition phase
\item[server] Generated by the data retrieval phase
\item[admin] Generated by an administrative procedure
\item[control] Generated by the controlling routines (usually after an error)
\item[insert] Generated by normal update routines (seldom seen)
\end{TTentry} \\

source\_archie\_hostname &
The name of the Archie server responsible for monitoring information at this
Data Host \\

primary\_ipaddr & 
The primary IP address of the Data Host used internally by the Archie system \\

access\_methods &
The name of the Archie catalog to which this data belongs. E.g., anonftp (for
anonymous ftp listings), webindex (for WWW pages) etc .. \\

access\_command &
The catalog-specific sequence of parameters used during the Data Acquisition
phase to perform the acquisition of the raw data from the Data Host. \\

os\_type & 
The operating system type of the Data Host. \\

timezone & 
The timezone of the Data Host. \\

retrieve\_time &
The time of data acquisition from the data host. This is written as
YYYYMMDDHHMMSS (year, month, day, hour, minute, second) and is always in UTC
(GMT). \\

parse\_time &
The time the data was parsed. Written in the same format as the retrieve\_time
field. \\

update\_time &
The time the data was updated. Written in the same format as the retrieve\_time
field. \\

no\_recs & 
The number of ``records'' in this data. For example, the value for a file
listing would be the number of files in the listing. This field is not used by
all catalogs. \\

current\_status &
Lists the current status of the data host. This can be:

\begin{TTentry}{del\_by\_archie}
\item[active]     available to be queried and inspected
\item[inactive]   temporarily disabled by the system
\item[del\_by\_archie]   scheduled to be deleted. Usually means that the data
in the system is out of date
\item[del\_by\_admin] scheduled to be deleted by the local administrator
\item[disabled] deactivated by the local Archie administrator
\item[not\_supported] catalog type is not supported at this data host
\end{TTentry} \\

update\_status & 
One of fail or succeed. Used internally by the system to determine result of
the previous phase of the update. \\

prospero\_host &
One of yes or no describing if the Prospero system is in operation at that
site. \\

data\_name & 
Name of individual data in current file. \\
\end{longtable}





\end{appendix}