%
% Copyright (c) 1996 Bunyip Information Systems Inc.
% All rights reserved
%
% Archie 3.5
% August 1996
%
% manual.tex
%

\chapter{This Manual}

\section{Who should use this manual}

Most of the information in this manual is prepared at a technical level. In
order to carry out the necessary tasks for maintaining an Archie installation,
you should have a reasonably strong technical background in UNIX system
administration. Thus, it is assumed you feel comfortable with the level of
material presented in the books listed at the end of this section.

\section{How to use this manual}

This manual has been divided into sections to deal with the major steps in
getting Archie out of the (virtual) box, and into the working world of your
system. The first section present an introduction to the Archie system;
subsequent sections deal with Archie installation and configuration.

\section{Font conventions}
In this document pathnames are represented in this \texttt{font}.
System manual page
titles and program names are in \textbf{bold typeface}.
Command line entries are
prefaced with a `\texttt{\%}' sign like this:

\comm{cat /etc/passwd}


New features in the version \version are indicated by the symbol presented
here in the margin. \new


\alertbox{Special notes are enclosed in boxes}



\NOTE Sections that start with the symbol displayed here in the margin
indicate troubleshooting information. These sections provide a checklist of
items to be resolved if trying to locate a problem within the system. Common
problems are listed, as are ways to fix them. 

\section{Suggested readings}

The following books describe UNIX systems/networks at the level of expertise
expected from an Archie system administrator. They provide excellent reference
material if you feel there is some area of expertise in which you require more
information.

\begin{itemize}
\item
The Whole Internet User's Guide and Catalog, Ed Krol, O'Reilly and
Associates, Inc., California, USA, 1992. (ISBN 1-56592-025-2)

\item
UNIX System Administration Handbook, Evi Nemeth, Garth Snyder, and Scott
Seebass, Prentice Hall, Englewood Cliffs, New Jersey, USA, 1989. (ISBN
0-13-933441-6)

\item
The UNIX Programming Environment, B. W. Kernighan and R. Pike, Prentice Hall,
Englewood Cliffs, New Jersey, USA, 1984.

\item
UNIX Network Programming, W. Richard Stevens, Prentice Hall, Englewood Cliffs,
New Jersey, USA, 1990. (ISBN 0-13-949876-1)

\item
Internetworking with TCP/IP Volume 1, Douglas E. Comer, Prentice Hall,
Englewood Cliffs, New Jersey, USA,1991. (ISBN 0-13-468505-9)

\item
Internetworking with TCP/IP Volume 3, Douglas E. Comer, Prentice Hall,
Englewood Cliffs, New Jersey, USA,1993. (ISBN 0-13-474222-2)
\end{itemize}



\section{Where to go from here}

The rest of this manual has been divided into sections to deal with the major
steps in getting Archie out of the box and into the working world of your
system. It is organized as follows:

\begin{itemize}

\item
``Installing the basic Archie system'' on page~\pageref{chap:install}
describes how to get the new
Archie software and initially installed on your system.

\item 
To properly configure the Archie system for your site, you must have an
understanding of the architecture of the Archie system; a discussion of this
is presented in ``System Overview'' on page~\pageref{chap:overview}.

\item
``Configuring the Basic System'' on page~\pageref{chap:configure}
deals with configuring your Archie
software to carry out the tasks you require at your site.
\end{itemize}

The concluding section, contains a list of Frequently Asked Questions about
the Archie system which you may like to use as your first reference when
something seems to go wrong.

